%%%%%%%%%%%%%%%%%%%%%%%%%%%%%%%%%%%%%%%%%
% Journal Article
% LaTeX Template
% Version 1.4 (15/5/16)
%
% This template has been downloaded from:
% http://www.LaTeXTemplates.com
%
% Original author:
% Frits Wenneker (http://www.howtotex.com) with extensive modifications by
% Vel (vel@LaTeXTemplates.com)
%
% License:
% CC BY-NC-SA 3.0 (http://creativecommons.org/licenses/by-nc-sa/3.0/)
%
%%%%%%%%%%%%%%%%%%%%%%%%%%%%%%%%%%%%%%%%%

%----------------------------------------------------------------------------------------
%	PACKAGES AND OTHER DOCUMENT CONFIGURATIONS
%----------------------------------------------------------------------------------------

\documentclass[twoside,twocolumn]{article}

\usepackage{blindtext} % Package to generate dummy text throughout this template 

\usepackage[sc]{mathpazo} % Use the Palatino font
\usepackage[T1]{fontenc} % Use 8-bit encoding that has 256 glyphs
\linespread{1.05} % Line spacing - Palatino needs more space between lines
\usepackage{microtype} % Slightly tweak font spacing for aesthetics

\usepackage[english]{babel} % Language hyphenation and typographical rules

\usepackage[hmarginratio=1:1,top=32mm,columnsep=20pt]{geometry} % Document margins
\usepackage[hang, small,labelfont=bf,up,textfont=it,up]{caption} % Custom captions under/above floats in tables or figures
\usepackage{booktabs} % Horizontal rules in tables

\usepackage{lettrine} % The lettrine is the first enlarged letter at the beginning of the text

\usepackage{enumitem} % Customized lists
\setlist[itemize]{noitemsep} % Make itemize lists more compact

\usepackage{abstract} % Allows abstract customization
\renewcommand{\abstractnamefont}{\normalfont\bfseries} % Set the "Abstract" text to bold
\renewcommand{\abstracttextfont}{\normalfont\small\itshape} % Set the abstract itself to small italic text

\usepackage{titlesec} % Allows customization of titles
\renewcommand\thesection{\Roman{section}} % Roman numerals for the sections
\renewcommand\thesubsection{\roman{subsection}} % roman numerals for subsections
\titleformat{\section}[block]{\large\scshape\centering}{\thesection.}{1em}{} % Change the look of the section titles
\titleformat{\subsection}[block]{\large}{\thesubsection.}{1em}{} % Change the look of the section titles

\usepackage{fancyhdr} % Headers and footers
\pagestyle{fancy} % All pages have headers and footers
\fancyhead{} % Blank out the default header
\fancyfoot{} % Blank out the default footer
\fancyhead[C]{Running title $\bullet$ May 2016 $\bullet$ Vol. XXI, No. 1} % Custom header text
\fancyfoot[RO,LE]{\thepage} % Custom footer text

\usepackage{titling} % Customizing the title section

\usepackage{hyperref} % For hyperlinks in the PDF

%----------------------------------------------------------------------------------------
%	TITLE SECTION
%----------------------------------------------------------------------------------------

\setlength{\droptitle}{-4\baselineskip} % Move the title up

\pretitle{\begin{center}\Huge\bfseries} % Article title formatting
\posttitle{\end{center}} % Article title closing formatting
\title{Motion planning of a fixed wings Uav through an hybrid approach based on artificial potential
  fields and RRT. } % Article title
\bigskip
\author{%
\textsc{Edoardo Ghini} \\[1ex] % Your name
\normalsize \href{mailto:ghiniedoardo@gmail.com}{ghiniedoardo@gmail.com} % Your email address
\and % Uncomment if 2 authors are required, duplicate these 4 lines if more
\textsc{Gianluca Cerilli} \\[1ex] % Second author's name
\normalsize \href{mailto:gianlucer@gmail.com}{gianlucer@gmail.com}\\ % Second author's email address
\normalsize Dipartimento di Ingegneria dell'Universita di Roma La Sapienza\\
}
\date{\today} % Leave empty to omit a date
\renewcommand{\maketitlehookd}{%
\begin{abstract}
\noindent  This is the abstract, try to be concise !
\end{abstract}
}

%----------------------------------------------------------------------------------------

\begin{document}

% Print the title
\maketitle

%----------------------------------------------------------------------------------------
%	ARTICLE CONTENTS
%----------------------------------------------------------------------------------------

\section{Introduction}

\lettrine[nindent=0em,lines=3]{W}e will present a work made
for an international challenge\footnote{The challenge is the AUVSI-SUAS hosted
  in united states in summer 2018.} that every year involves several accademic teams.\\
This article has the following structure : \\
in section 2, we will present the problem statement and in section 3 
 we will give tecnical informations on the hardware at our disposal,
finally in section 4, the adopted solution will be discussed.
 
%------------------------------------------------

\section{Problem Statement}
Maecenas sed ultricies felis. Sed imperdiet dictum arcu a egestas. 
\section{Hardware components}
\subsection{Structural design}
Maecenas sed ultricies felis. Sed imperdiet dictum arcu a egestas. 
\subsection{On board computing}
Maecenas sed ultricies felis. Sed imperdiet dictum arcu a egestas. 
\subsection{Sensing instruments}
Maecenas sed ultricies felis. Sed imperdiet dictum arcu a egestas. 
\subsection{Communication devices}
Maecenas sed ultricies felis. Sed imperdiet dictum arcu a egestas. 
\subsection{Auto pilot framework}
Maecenas sed ultricies felis. Sed imperdiet dictum arcu a egestas. 
\section{Hybrid planner}

\subsection{RRT}
Maecenas sed ultricies felis. Sed imperdiet dictum arcu a egestas. 
\subsection{Artificial potentials}
Maecenas sed ultricies felis. Sed imperdiet dictum arcu a egestas. 
\subsection{Implementation}
Maecenas sed ultricies felis. Sed imperdiet dictum arcu a egestas. 

%%\footnote{Example footnote}.

%------------------------------------------------

\section{Conclusion}

% \begin{table}
% \caption{Example table}
% \centering
% \begin{tabular}{llr}
% \toprule
% \multicolumn{2}{c}{Name} \\
% \cmidrule(r){1-2}
% First name & Last Name & Grade \\
% \midrule
% John & Doe & $7.5$ \\
% Richard & Miles & $2$ \\
% \bottomrule
% \end{tabular}
% \end{table}


\begin{equation}
\label{eq:emc}
e = mc^2
\end{equation}


%----------------------------------------------------------------------------------------
%	REFERENCE LIST
%----------------------------------------------------------------------------------------

\begin{thebibliography}{99} % Bibliography - this is intentionally simple in this template

\bibitem[Figueredo and Wolf, 2009]{Figueredo:2009dg}
Figueredo, A.~J. and Wolf, P. S.~A. (2009).
\newblock Assortative pairing and life history strategy - a cross-cultural
  study.
\newblock {\em Human Nature}, 20:317--330.
 
\end{thebibliography}

%----------------------------------------------------------------------------------------

\end{document}
