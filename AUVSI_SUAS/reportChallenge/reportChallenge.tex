\documentclass[oneside,onecolumn]{article}


\usepackage{blindtext} % Package to generate dummy text throughout this template 

\usepackage[sc]{mathpazo} % Use the Palatino font
\usepackage[T1]{fontenc} % Use 8-bit encoding that has 256 glyphs
\linespread{1.05} % Line spacing - Palatino needs more space between lines
\usepackage{microtype} % Slightly tweak font spacing for aesthetics

\usepackage[english]{babel} % Language hyphenation and typographical rules

\usepackage[hmarginratio=1:1,top=32mm,columnsep=20pt]{geometry} % Document margins
\usepackage[hang, small,labelfont=bf,up,textfont=it,up]{caption} % Custom captions under/above floats in tables or figures
\usepackage{booktabs} % Horizontal rules in tables

\usepackage{lettrine} % The lettrine is the first enlarged letter at the beginning of the text

\usepackage{enumitem} % Customized lists
\setlist[itemize]{noitemsep} % Make itemize lists more compact

\usepackage{abstract} % Allows abstract customization
\renewcommand{\abstractnamefont}{\normalfont\bfseries} % Set the "Abstract" text to bold
\renewcommand{\abstracttextfont}{\normalfont\small\itshape} % Set the abstract itself to small italic text

\usepackage{titlesec} % Allows customization of titles
\renewcommand\thesection{\Roman{section}} % Roman numerals for the sections
\renewcommand\thesubsection{\roman{subsection}} % roman numerals for subsections
\titleformat{\section}[block]{\large\scshape\centering}{\thesection.}{1em}{} % Change the look of the section titles
\titleformat{\subsection}[block]{\large}{\thesubsection.}{1em}{} % Change the look of the section titles

\usepackage{fancyhdr} % Headers and footers
\pagestyle{fancy} % All pages have headers and footers
\fancyhead{} % Blank out the default header
\fancyfoot{} % Blank out the default footer
% \fancyhead[C]{Running title $\bullet$ May 2016 $\bullet$ Vol. XXI, No. 1} % Custom header text
\fancyfoot[RO,L]{\thepage} % Custom footer text

\usepackage{titling} % Customizing the title section

\usepackage{hyperref} % For hyperlinks in the PDF

\usepackage{romannum} % Roman numbers 
\usepackage{graphicx}
\usepackage{wrapfig}
\graphicspath{ {images/} }
\usepackage{wrapfig}
\usepackage{subcaption}

% ----------------------------------------------------------------------------------------
%	TITLE SECTION
% ----------------------------------------------------------------------------------------

\setlength{\droptitle}{-4\baselineskip} % Move the title up

\pretitle{\begin{center}\Huge\bfseries} % Article title formatting
  \posttitle{\end{center}} % Article title closing formatting
\title{Motion planning of a fixed wings Uav through an hybrid approach based on artificial potential
  fields and RRT. } % Article title
\author{%
  \textsc{Edoardo Ghini} \\[1ex] % Your name
  \normalsize \href{mailto:ghiniedoardo@gmail.com}{ghiniedoardo@gmail.com} % Your email address
  \and % Uncomment if 2 authors are required, duplicate these 4 lines if more
  \textsc{Gianluca Cerilli} \\[1ex] % Second author's name
  \normalsize \href{mailto:gianlucer@gmail.com}{gianlucer@gmail.com}\\ % Second author's email address
  \normalsize Dipartimento di Ingegneria dell'Universita di Roma La Sapienza\\
}
\date{\today} % Leave empty to omit a date
% \renewcommand{\maketitlehookd}{%
% \begin{abstract}
%   \noindent  This is the abstract, try to be concise !
% \end{abstract}
% }
% ----------------------------------------------------------------------------------------

\begin{document}

% Print the title
\maketitle
\bigskip
\bigskip
\bigskip
\bigskip
\begin{center}
  \includegraphics[width=0.3\textwidth]{laSapienza}
\end{center}


% ----------------------------------------------------------------------------------------
%	ARTICLE CONTENTS
% ----------------------------------------------------------------------------------------
\newpage
\section{Introduction}

\lettrine[nindent=0em,lines=3]{T}his paper presents a work made
for an international challenge\footnote{The challenge is the AUVSI-SUAS hosted in united states in summer 2018.} that every year involves several academic teams.\\
The following structure has been respected: \\
in section \Romannum{2}, the problem statement is presented; section \Romannum{3} describes technical informations on the hardware and finally in section \Romannum{4}, the adopted solution will be discussed.

% ------------------------------------------------

\section{Problem Statement}
All the teams joining the challenge have to compete in several tasks concerning actual problems about the governance of an Unmanned Aerial Vehicle.\\
Each team brings its prototype of UAV on a common flight ground and tries to score the greatest number of points among all the tasks proposed, which are:
\begin{enumerate}\centering
\item Autonomous Flight
\item Obstacle Avoidance
\item Object Detection
\item Object Classification
\item Object Localization 
\item Air Delivery
\end{enumerate}
This will be the first partecipation at a competition of this kind for the La Sapienza team, indeed the previous challenges in which the University has already been involved required a human controlled guidance system. Therefore, the autonomous flight issue has to be faced without previous
insights on the matter.
\newpage

\subsection{Starting Point}
The aeronautical research department, in the past decade, has spent a big effort working on an autonomous guidance system, that has been designed and implemented by Master students in their thesis.\\
This system allows to control the complex aerodynamics of the vehicle from an higher level of abstraction. Mainly, it uses a series of waypoints that describes a trajectory.\\\\
\begin{wrapfigure}{r}{5.5cm}
\caption{YAK scaled aero model}\label{wrap-fig:1}
\includegraphics[width=5.5cm]{YAK1}
\end{wrapfigure} 
In this way the aircraft will likely follow a path made of lines that intersects subsequent waypoints.\\
The system is designed to work on an on-board computer and to communicate with a ground station that continuously sends to and receive from the judges' server, telemetrical data and mission objectives.\\
A fixed-wings radio controlled aircraft model\footnote{It is a scaled reproduction (1:3 ratio) of the YAK 112} has been chosen, like those commonly used in hobby modeling, as shown in Figure.~\ref{wrap-fig:1}

\subsection{Main task}
This paper describes the work concerning the area of \textbf{automation} and \textbf{robotics} of an atomic part of the full Sapienza Flight team, which counts several members coming from an aeronutical background.\\ So, the autonomous guidance problem that comprehends also the obstacle avoidance clause, has been engaged.\\
The competition is organized such that, during the various missions of the challenge, each team receives pose informations about fixed and mobile \textbf{virtual} obstacles.\\
Then, the judges check the success or failure in avoiding obstacles according to the \textbf{telemetry} data that they receive from each team.\\
So, there the main goal was to design a \textbf{path planning} algorithm that could bring the UAV from a start to a goal position without collisions.

\section{Hardware components}

\subsection{Propulsion}
\begin{wrapfigure}{r}{5.5cm}
\caption{DLE55 motor}\label{wrap-fig:2}
\includegraphics[width=5.5cm]{YAK3}
\end{wrapfigure} 

The engine chosen is a \textbf{DLE55} (in Figure~\ref{wrap-fig:2} ) :

\begin{enumerate}
\item 50 cc monocylinder petrol engine
\item reservoir capable of containing 950cc of petrol
\item equipped with a 7, 4V LiPo battery
\end{enumerate}


\subsection{On board computing}
The autonomous guidance architecture, implemented in Matlab and Simulink, is converted in compiled high speed C, Cpp language. It is run by Arduino Due and there is also a Raspberry pi Model 3 that runs a kalman filter in
order to process all signals from sensors.

% TODO at the end we will focus on layouts and image positions

\subsection{Sensing instruments}

\subsection{Communication devices}
 
\subsection{Auto pilot framework}

\section{Hybrid planner}
Two different strategies have been merged, since both present different shortcomings and advantages.

\subsection{RRT}
RRT (Rapidly-exploring Random Tree) is a probabilistic planner that randomly builds a space-building tree.\\
It has been used offline considering some motion primitives (produced by some specific velocity inputs) to define a path for the UAV, biasing it towards the unexplored areas closest to the goal.
\subsection{Artificial potentials}
Artificial Potential Fields (APF) have been considered for two main reasons:
\begin{itemize}
	\item the world can be approximated to a "world of spheres" (since there are cylindrical obstacles that we consider in 2D), that prevents the UAV entering the basin of attraction of some local minima
	\item when the UAV faces an obstacle, RRT must expand many branches before getting around it. This leads to a significant slowdown
\end{itemize}
Using APF the UAV can react as soon as entering the range of influence of the obstacle, with a smooth movement. This also because an implementation with vertex fields has been made, replacing repulsive actions with actions forcing the robot to get around it.
\subsection{Implementation}
We consider the kinematic model of a fixed-wing UAV flying at a constant altitude. So, we can neglect the pitch angle, obtaining:
\begin{equation}
\begin{array} {lcl} 
\dot{x} & = & vcos\psi \\
\dot{y} & = & vsin\psi \\ 
\dot{\psi} & = & -\frac{g}{v} tan \phi \\
\dot{\phi} & = & u_{\phi }
\end{array}
\label{linearsystem}
\end{equation}
in which the speed $v$ and roll rate $u_{\phi}$ are the control inputs.

As described above, for the path planning we use a mixed approach between RRT and Artificial Potential Fields methods.\\
At the beginning, a random value is generated. If this value is greater or less than a factor (that is decremented after each cycle), the RRT generates respectively a position biased to the goal or a random one, depending from the five primitives of the UAV and it expands a node that represents the closest possible position of the UAV to the new confguration. If this new position collides with an obstacle, the node is not added to the tree and becomes unavailble for next nodes generation (in a similar way, if all the children of a parent collide with some obstacle, both the children and the parent become unavailable).
These five primitives are generated and ordered according to another factor obtained from the 
APF method. Particularly, depending from the configuration of the UAV and from the range of influence of the obstacles, the total vortex field acting on the UAV is computed. 
\begin{figure}[htbp]
	\centering
	\includegraphics[width=6.5cm]{images/b}%
	\qquad\qquad
	\includegraphics[width=6.5cm]{images/a}\label{fig:3}
	\caption{First image: Positions generated with vortex fields in red; Second image: Final path}
\end{figure}

Since the resultant vector of the vortex fields is directed toward the goal, we compute the direction of each primitive and select the one that forms the smallest angles with the direction of the vortex. Thanks to the vortex fields the UAV get around the obstacles avoiding local minima and proceeds faster the goal.
\begin{figure}[htbp]
	\centering
	\includegraphics[width=6.5cm]{images/c}%
	\qquad\qquad
	\includegraphics[width=6.5cm]{images/d}\label{fig:4}
	\caption{First image: Repulsive fields; Second image: Vortex fields (tangent)}
\end{figure}


%% \footnote{Example footnote}.

% ------------------------------------------------

\section{Conclusion}

% \begin{table}
%   \caption{Example table}
%   \centering
%   \begin{tabular}{llr}
%     \toprule
%     \multicolumn{2}{c}{Name} \\
%     \cmidrule(r){1-2}
%     First name & Last Name & Grade \\
%     \midrule
%     John & Doe & $7.5$ \\
%     Richard & Miles & $2$ \\
%     \bottomrule
%   \end{tabular}
% \end{table}


\begin{equation}
  \label{eq:emc}
  e = mc^2
\end{equation}


% ----------------------------------------------------------------------------------------
% REFERENCE LIST
% ----------------------------------------------------------------------------------------

\begin{thebibliography}{99} % Bibliography - this is intentionally simple in this template

\bibitem[Figueredo and Wolf, 2009]{Figueredo:2009dg}
  Figueredo, A.~J. and Wolf, P. S.~A. (2009).
  \newblock Assortative pairing and life history strategy - a cross-cultural
  study.
  \newblock {\em Human Nature}, 20:317--330.
  
\end{thebibliography}

% ----------------------------------------------------------------------------------------

\end{document}
